%!TEX root = mieic-en.tex

\chapter{Revisão Bibliográfica} \label{chap:sota}

\section*{}
\textit{
Neste capítulo é descrito o estado da arte e são
apresentados trabalhos relacionados para mostrar o que existe no
mesmo domínio e quais os problemas em aberto.
Deve deixar claro que existe uma oportunidade de desenvolvimento que
cobre alguma falha concreta .
O capítulo deve também efetuar uma revisão tecnológica às principais
ferramentas utilizáveis no âmbito do projeto, justificando futuras
escolhas.
}

Resenha utilizaçao computer vision em trafego e em transportes
Concluir com os desafios da armis (drive, quere aplicaro computer vision para automatizar determinados aspetos)
Sequência de funcionalidades que respondem ao desafio da Armis(segmentaçao,deteçao,...)
Uma secção para cada um

Tese Gustavo Lira / Pedro Loureiro

\section{History of Computer Vision}

Computer vision appeared as an area of investigation around the mid 60's at the MIT by Professor Larry Roberts, whose PhD. thesis focused on methods to extract 3D information from a 2D image ~\cite{huang_computer_1996} in order to reconstruct entire scenes from the geometry gathered. This area is now considered by the ACM as a branch of artificial intelligence according to the 2012 ACM Computing Classification System ~\cite{acm_2012_2012}. From this point onward scientists began applying the techniques developed in multiple fields.

The use of computer vision in manufacturing industry was among the first to be explored, as there were already multiple robots working in the assembly lines of the factories that needed to be improved, as noted by Stout in 1980 ~\cite{stout_computer_1980}. These robots were becoming outdated due to the lack of interaction with the environment. This meant that for a robot to work with a part in an assembly line it needed to be placed in a pre-determined position, and any deviation could mean that the line needed to be stopped. Michael Baird relates in ~\cite{l._baird_sight-i:_1978} a computer vision system developed to inspect circuit chips rotation in a welding base. Computer vision was also used as a tool to automatically analyse weather satellite images ~\cite{binford_computer_1973-2} and multidimensional medical images ~\cite{ayache_medical_1998}, with applications being developed in these areas being now common in our daily life.

Regarding intelligent transport systems (ITSs), the use of computer vision to aid in the analysis of traffic has been increasing in the last years due to the decreasing costs of hardware, both cameras, storage and processing power, as well as the knowledge to extract useful data from the video gathered. One of the first works applying computer vision to analyse traffic was published in 1984 ~\cite{dickinson_image_1984}

Since the dawn of computer vision in the early 90's as an area of artificial intelligence, according to the  ~\cite{acm_2012_2012}, that 

\section{Segmentation}
Image segmentation is the process through which  an image is separated into its different regions according to the desired output, usually objects of interest. These regions are composed by pixels that have a common characteristic~\cite{shapiro_computer_2001}


\section{Secção Exemplo}\label{sec:dialecto}

\emph{Scalable Vector Graphics}\index{SVG}\index{XML!SVG} é uma
linguagem em formato XML que descreve gráficos de duas dimensões. 
Este formato padronizado pela W3C (\emph{World Wide Web Consortium})
é livre de patentes ou direitos de autor e está totalmente
documentado, à semelhança de outros W3C
standards~cite{kn:svgdoc}.

Sendo uma linguagem XML, o \svg{} herda uma série de vantagens: a
possibilidade de transformar \svg{} usando técnicas como
XSLT\index{XML!XSLT}, de embeber \svg{} em qualquer documento
XML\index{XML} usando \textit{namespaces} ou até de  
estilizar \svg{} recorrendo a CSS\index{CSS} (\emph{Cascade Style Sheets}). 
De uma forma geral, pode dizer-se que \svg{}s interagem bem com as
atuais tecnologias ligadas ao XML e à Web, tal como referido
em~cite{kn:svgibm,kn:svgw3c}.

\section{Resumo ou Conclusões}

No final do capítulo deverá ser apresentado um resumo com as 
principais conclusões que se podem tirar. 

Vivamus non nunc nec risus tempor varius. Quisque bibendum mi at
dolor. Aliquam consectetuer condimentum risus. Aliquam luctus pulvinar
sem. Duis aliquam, urna et vulputate tristique, dui elit aliquet nibh,
vel dignissim magna turpis id sapien. Duis commodo sem id
quam. Phasellus dolor. Class aptent taciti sociosqu ad litora torquent
per conubia nostra, per inceptos himenaeos. 
