\chapter*{Resumo}
A gestão de tráfego é uma das tarefas mais cruciais e importantes em cidades modernas, dado o crescente número de veículos nas sub-dimensionadas redes viárias. Para apoiar esta tarefe existem multiplas soluções que simulam o comportamento das redes, que usam como fonte de dados métodos de contagem tradicionais tais como as espiras de indução. Estes dispositivos são defeituosos e limitados, assim como bastante dispendiosos na instalação e manutenção. Com o decrescente custo de câmaras de vigilância e o estado da visão por computador, o uso de ambos tornou-se uma alternativa viável para a análise de tráfego.

Neste projeto abordamos o problema de vigilância automática de tráfego utilizando câmaras de video, desenvolvendo um processador de analíticas de video para realizar tarfeas de deteção e classificação nos dados recolhidos. Este processador foi integrado numa solução em desenvolvimento que processa videos de múltiplas fontes e disponibliza visão por computador como um serviço online.

Uma nova abordagem para o problema da classificação de veículos é apresentado, baseado no uso de conjuntos difusos. Para ilustrar a abordagem proposta, a deteção e classificação implementadas foram testadas com diferentes câmaras, mostrando resultados promissores.

\chapter*{Abstract}
Traffic management is one of the most important and crucial tasks in modern cities, due to the increasing number of vehicles in the under dimensioned road networks. To aid in this task multiple solutions exist that simulate the networks behaviour taking for input data from traditional counting methods such as magnetic induction loops. These devices present flaws and are expensive to implement. With the decreasing cost of video cameras and the wide-spread use of computer vision, their use became a viable alternative to analyse traffic.

In this project we tackle the problem of automatic traffic surveillance using video cameras, developing a video analytics processor that will perform detection and classification tasks over gathered data. This processor was integrated in a solution under development that processes videos from multiple sources and provides computer vision as an on-line service.

A novel approach to the vehicle classification process is presented, based on the use of a fuzzy set. To illustrate the proposed approach, the detection and classification implemented were tested with different cameras in different scenarios, showing promising results. 
