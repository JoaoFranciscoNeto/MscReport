\chapter{Implementation}\label{chap:chap4}
\listoftodos
\section*{}

\textit{
Este capítulo pode ser dedicado à apresentação de detalhes de nível
mais baixo relacionados com o enquadramento e implementação das
soluções preconizadas no capítulo anterior.
Note-se no entanto que detalhes desnecessários à compreensão do
trabalho devem ser remetidos para anexos.
Dependendo do volume, a avaliação do trabalho pode ser incluída neste
capítulo ou pode constituir um capítulo separado.
}


As previously stated, the work done for this project will consist of a module for the "Video Server" being developed by LIACC, that will perform all the required analytics of the video received. This chapter provides the details of the implementation of this module, developed during the dissertation semester between February and June 2017. Tasks performed involved designing an appropriate architecture, able to scale with the project, treating the images received from the video streams to detect events and extract information.

It is expected that this chapter provides some insight on how the theoretical backgrounds presented before were used during the project development, namely phase 2 and 3 of the presented project planning\missingfigure{Project Planning}. Phase 2 focused on the detection of relevant events to the traffic controller such as suspicious approximation to the camera, intrusion in prohibited areas, fallen objects on the road and wrong way travelling, while phase 3 aimed to count and classify vehicles in both urban and high-way scenarios.

\section{Technology}

\subsection{OpenCV}
OpenCV is a library composed of implementations of useful computer vision algorithms implementation, widely used across the industry and academy. It has interfaces for multiple programming languages, C++, Java and Python, but is natively written in C++ in order to take advantage of low level performance enhancements, as performance is an important factor in real time computer vision applications.
The library contains over 2500 algorithms ranging from the more basic image processing such as filtering, morphology operators and geometric transformations to more complex ones that are able to, among others, compare images, track features and camera movements and recognize faces. Along with the image processing capabilities OpenCV also ships with interfaces to stereo cameras such as Kinect that allow users to retrieve a cloud of 3D points and a depth map from the captured image. This was the chosen library as there are researchers at LIACC that had already worked with it and provided invaluable guidance during the various development stages.

\subsection{JavaCV}
JavaCV is a wrapper for OpenCV written in Java that works on top of the JavaCPP Presets, a project that provides Java interfaces for commonly used C++ libraries, such as OpenCV and FFmpeg, the ones we are using, as well as CUDA, ARToolKitPlus and others. It provides access to all the functionalities of OpenCV inside a Java environment, and was the chosen solution as there was already experience inside LIACC working with this technology.

\section{Architecture}

\missingfigure{Diagrama Arquitetura}

The analytics module was designed to comply with the following specifications, imposed by both our partner and to comply with the already developed modules. 

\begin{itemize}
	\item Run uninterruptedly waiting for requests to be made
	\item Process a large number of video inputs at the same time
	\item Receive input from streams or files
	\item Specify which analysis are to be run on a specific video
	\item
\end{itemize}
\todo{Mais requirements?}

The project main thread runs an instance of the AnalyticsManager class that implements the Java Runnable interface. This thread will launch one CameraAnalyticsManager instance for each video to be analysed, be it from stream or from file, and keep track of each worker status, disposing of them when they finish, and launching new ones when a request is received. The CameraAnalyticsManager is then responsible to query the database in order to find out what are the analytics the user wants to retrieve from the video, information that is stored in the \textit{camera} table of the database. This process then starts a frame grabber that will convert each frame of the video into it's representation in OpenCV and feed it to each Analyser through a queue. This enables each one of these workers to run at his own pace and if one of them runs slower than the pace at which the frame grabber reads the images, it will simply queue them up and not slow down the remaining workers. The drawbacks of this solution are that it is theoretically possible to run out of memory to keep these frames, although this limit was not reached during the testing phase.

\section{Secção Exemplo}

%\todofigure{Inserir uma figura sobre o Map/Reduce}

Lorem ipsum dolor sit amet, consectetuer adipiscing elit. Integer
hendrerit commodo ante. Pellentesque nibh libero, aliquam at, faucibus
id, commodo a, velit. 
%\todoline{Escrever sobre o map/reduce}
Duis eleifend sem eget leo. Morbi in est. Suspendisse magna sem,
varius nec, hendrerit non, tincidunt quis, quam. Aenean congue. 
%\todolines{A short entry in the list of todos}{A very long todonote
%  that certainly will fill more than a single line in the list of
%  todos. Just to make sure let's add some more text.} 
Vivamus vel est sit amet sem iaculis posuere. Cras mollis, enim vel
gravida aliquam, libero nunc ullamcorper dui, ullamcorper sodales
lectus nulla sed urna. Morbi aliquet porta risus. 
Proin vestibulum ligula a purus. Maecenas a nulla. 
Maecenas mattis est vitae neque auctor tempus. Etiam nulla dui,
mattis vitae, porttitor sed, aliquet ut, enim. Cras nisl magna,
aliquet et, laoreet at, gravida ac, neque. Sed id est. Nulla dapibus
dolor quis ipsum rhoncus cursus. 

\section{Mais uma Secção}

Lorem ipsum dolor sit amet, consectetuer adipiscing elit. Quisque
purus sapien, interdum ut, vestibulum a, accumsan ullamcorper,
erat. Mauris a magna ut leo porta imperdiet. Donec dui odio, porta in,
pretium non, semper quis, orci. Quisque erat diam, pharetra vel,
laoreet ac, hendrerit vel, enim. Donec tristique luctus risus. Fusce
dolor est, eleifend id, elementum sit amet, varius vitae, neque. Morbi
at augue. Ut sem ligula, auctor vitae, facilisis id, pharetra non,
lectus. Nulla lacus augue, aliquam eget, sollicitudin sed, hendrerit
eu, leo. Suspendisse ac tortor. Mauris at odio. Etiam vehicula. Nam
lacinia purus at nibh. Aliquam fringilla lorem ac justo. Ut nec
enim. 
%\todoref{Citar Map/reduce}

Quisque ullamcorper. Aliquam vel magna. Sed pulvinar dictum
ligula. Sed ultrices dolor ut turpis. Vivamus sagittis orci malesuada
arcu venenatis auctor. Proin vehicula pharetra urna. Aliquam egestas
nunc quis nisl. Donec ullamcorper. Nulla purus. Ut suscipit lacus
vitae dui. Mauris semper. Ut eget sem. Integer orci. Nam vitae dui
eget nisi placerat convallis. 

\begin{lstlisting}[float,language=Java, label=src:mapreduce, caption=Example map and reduce functions for word counting]
map(String key, String value): 
// key: document name 
// value: document contents 
for each word w in value:
EmitIntermediate(w, "1");

reduce(String key, Iterator values):
// key: a word 
// values: a list of counts 
int result = 0;
for each v in values: 
result += ParseInt(v);

Emit(AsString(result))
\end{lstlisting}

Sed id lorem. Proin gravida bibendum lacus. Sed molestie, urna quis
euismod laoreet, diam dolor dictum diam, vitae consectetuer leo ipsum
id ante. Integer eu lectus non mauris pharetra viverra. In feugiat
libero ut massa. Morbi cursus, lorem sollicitudin blandit semper,
felis magna pellentesque lacus, ut rhoncus leo neque at tellus. Sed
mattis, diam eget eleifend tincidunt, ligula eros tincidunt diam,
vitae auctor turpis est vel nunc. In eu magna. Donec dolor metus,
egestas sit amet, ultrices in, faucibus sed, lectus. Etiam est enim,
vehicula pharetra, porta non, viverra vel, nunc. Ut non sem. Etiam nec
neque. 

\section{Resumo ou Conclusões}

Proin vehicula pharetra urna. Aliquam egestas
nunc quis nisl. Donec ullamcorper. Nulla purus. Ut suscipit lacus
vitae dui. Mauris semper. Ut eget sem. Integer orci. Nam vitae dui
eget nisi placerat convallis. 
