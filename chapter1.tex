%!TEX root = mieic-en.tex

\chapter{Introdução} \label{chap:intro}

\section*{}

\textit{O primeiro capítulo da dissertação deve servir para apresentar o
enquadramento e a moti\-va\-ção do trabalho e para identificar e
definir os problemas que a dissertação aborda.
Deve resumir as metodologias utilizadas no trabalho e termina
apresentando um breve resumo de cada um dos capítulos
posteriores.}

\section{Contexto/Enquadramento} \label{sec:context}
\textit{
Esta secção descreve a área em que o trabalho se insere, podendo
referir um eventual projeto de que faz parte e apresentar uma breve
descrição da empresa onde o trabalho decorreu.
}



\section{Projeto} \label{sec:proj}

\textit{
Na continuação da secção anterior, e apenas no caso de ser um Projeto
e não uma Dissertação, esta secção apresenta resumidamente o projeto.
}


\section{Motivation and Goals} \label{sec:goals}

\textit{
Apresenta a motivação e enumera os objetivos do trabalho terminando
com um resumo das metodologias para a prossecução dos objetivos.
}

\section{Estrutura da Dissertação} \label{sec:struct}

Para além da introdução, esta dissertação contém mais x capítulos.
No capítulo~\ref{chap:sota}, é descrito o estado da arte e são
apresentados trabalhos relacionados. 
%\todoline{Complete the document structure.}
No capítulo~\ref{chap:chap3}, ipsum dolor sit amet, consectetuer
adipiscing elit.
No capítulo~\ref{chap:chap4} praesent sit amet sem. 
No capítulo~\ref{chap:concl}  posuere, ante non tristique
consectetuer, dui elit scelerisque augue, eu vehicula nibh nisi ac
est. 
