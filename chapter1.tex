%!TEX root = mieic-en.tex

\chapter{Introdução} \label{chap:intro}

\section*{}

\textit{O primeiro capítulo da dissertação deve servir para apresentar o
enquadramento e a moti\-va\-ção do trabalho e para identificar e
definir os problemas que a dissertação aborda.
Deve resumir as metodologias utilizadas no trabalho e termina
apresentando um breve resumo de cada um dos capítulos
posteriores.}

\section{Contexto/Enquadramento} \label{sec:context}
\textit{
Esta secção descreve a área em que o trabalho se insere, podendo
referir um eventual projeto de que faz parte e apresentar uma breve
descrição da empresa onde o trabalho decorreu.
}

Traffic management in cities is a vast area as it studies the planning and control of the road network. As the cities tend to evolve following the concept of \textit{smart cities}, this is also a field where the information technologies are being applied to improve the effectiveness of these tasks.

With the increase in the number of vehicles using the roads ~\cite{navigant_transportation_2017}, a need to improve the methods of traffic management rose as well. The development of multiple approaches to this issue represents both it's importance and the pertinence of the work being developed at LIACC in regards to this topic.

Evolving the existing solutions involves a process of reformulation of the methods being currently used. With the decreasing costs of cameras for video surveillance, the number of units installed around the world is rising, passing the 245 million mark in 2014 ~\cite{jenkins_245_2015}, over 65\% of that number being from Asia.

With this number of cameras placed globally the amount of data being collected every day is too large to be humanly processed, and thus the need to create autonomous processors arises. The market for automatic analysis of video is expected to be worth 11.17 Billion USD by 2022 ~\cite{reportlinker_global_2017}, with the facial recognition area expected to have the highest CAGR (Compound Annual Growth Rate) due to the potential related to the security applications, even going as far as replacing airport checks in Australia by 2020 ~\cite{koziol_world_2017}.

The usage of computer vision to analyse traffic conditions has been around for some time now, with earlier works dating to 1990, such as the work by Rourke et. al ~\cite{rourke_road_1990} where the advantages of using video analysis in this area are listed, as well as some issues that the community is still facing. Since then the field has been explored by many authors, applied to applications ranging from vehicle counting ~\cite{coifman_real-time_1998} ~\cite{beymer_real-time_1997} to incidents detection. 

Armis-Group is a company based in Porto that builds software in the areas of Information Management, Digital Sports and Intelligent Transport Systems.

\section{Project} \label{sec:proj}

\textit{
Na continuação da secção anterior, e apenas no caso de ser um Projeto
e não uma Dissertação, esta secção apresenta resumidamente o projeto.
}

The project for this dissertation was part of a collaboration between LIACC and Armis-Group. This collaboration aims to develop solutions in the field of Traffic Management, taking advantage of the laboratory's resources and the company's expertise and experience in the area. The objective of the project was to design and develop a software module capable of analysing video and extracting relevant events on the roads as well as performing counting and basic classification of vehicles. 

In the scope of the mentioned collaboration, LIACC developed a set of applications that form the "Video Server", a software package that 

\section{Motivation and Goals} \label{sec:goals}

\textit{
Apresenta a motivação e enumera os objetivos do trabalho terminando
com um resumo das metodologias para a prossecução dos objetivos.
}

The main goal of the dissertation is then to implement a solution that fits the needs of the project as stated above, detecting events in a short window of time to allow for fast response from the authorities, extracting 

This goal can be broke down into more specific tasks to allow for a better understanding of the work ahead:

\begin{itemize}
	\item Explore state of the art techniques that address the multiple obstacles to be faced along the process;
	\item Work in close relationship with the collaborators of the project;
	\item Implement solutions to perform the following tasks:
		\subitem Clear the image of noise and other unwanted features;
		\subitem Find objects of interest;
		\subitem Classify detected objects;
		\subitem Detect relevant events.
	\item Testing of the implementation in a real scenario.
\end{itemize}



\section{Estrutura da Dissertação} \label{sec:struct}

